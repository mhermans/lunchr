\documentclass{beamer}
\usetheme{Warsaw}

\setbeamercovered{dynamic}
\usepackage{Sweave}
\begin{document}





%%%%%%%%%%%%%%%%%%%%%%%%%%%%%%%
% SLIDE 0

\section{Voorbereiding}

\begin{frame}{R Studio}

\begin{itemize}
    \item R Studio: omgeving voor "ruwe" R (IDE)
    \item http://www.rstudio.com/
\end{itemize}

\end{frame}


%%%%%%%%%%%%%%%%%%%%%%%%%%%%%%%
% SLIDE 1

\section{Inleiding}
\subsection{Motivatie}
\begin{frame}{Inleiding}

\begin{itemize}
    \item<1-> Waarom is R interessant? 
    \item<2-> R vs SAS
        \begin{itemize}
        \item<3-> In principe kunnen ze hetzelfde, maar... 
        \item<4-> In R kan je functies combineren, in SAS niet (= enorme flexibiliteit)
        
        % Voorbeeld geven? 
        
        \item<5-> Uitstekende grafische mogelijkheden (wordt zeker een topic)
        
        % Ook hier een voorbeeld?
        
        \end{itemize}
    \item<6-> R wordt een belangrijke tool, vooral binnen de wetenschap
    \item<7-> Leuke extratjes (rapporten, presentaties... )
\end{itemize}

\end{frame}

%%%%%%%%%%%%%%%%%%%%%%%%%%%%%%%
% SLIDE 2

\section{Datastructuren in R}

\subsection{Overzicht datastructuren}

\begin{frame}{Overzicht}

R heeft meerdere data structuren: 

\begin{itemize}
  \item Vector
  \item Matrix \& Array
  \item List
  \item Data frame (cfr SAS \& SPSS)
  \item Factor \& table
\end{itemize}

\end{frame}

%%%%%%%%%%%%%%%%%%%%%%%%%%%%%%%
% SLIDE 3

\subsection{Vectoren}

\begin{frame}{Definitie en types}

\begin{itemize}
  \item Vector: set van elementen van dezelfde soort (\textit{mode}, type variabele)
  \item Belangrijkste datasoorten: 
  
    \begin{itemize}
    \item Integer
    \item Numeric 
    \item Logical (Boolean)
    \item Character
    \end{itemize}
  \item Vergelijkbaar met een SAS-variabele, maar veel ruimer gebruik!
\end{itemize}

\end{frame}

%%%%%%%%%%%%%%%%%%%%%%%%%%%%%%%
% SLIDE 4


\begin{frame}[fragile]{Voorbeelden vectoren}



\begin{itemize}

\item
Voorbeeld \textit{numeric vector}
\begin{Schunk}
\begin{Soutput}
[1]  0.2  6.1  9.3 -4.4  6.7
\end{Soutput}
\end{Schunk}
\vspace{30pt}
\item
Voorbeeld \textit{character vector}
\begin{Schunk}
\begin{Soutput}
[1] "een"         "twee"        "spaties ook"
\end{Soutput}
\end{Schunk}
\vspace{30pt}
\item
Voorbeeld \textit{boolean vector}
\begin{Schunk}
\begin{Soutput}
[1]  TRUE FALSE FALSE  TRUE
\end{Soutput}
\end{Schunk}
\end{itemize}
\end{frame}

%%%%%%%%%%%%%%%%%%%%%%%%%%%%%%%
% SLIDE 5

\begin{frame}[fragile]{Niet zomaar een dataset variabele}

Onze numerieke vector... 
\begin{Schunk}
\begin{Sinput}
> print(numeric.v)
\end{Sinput}
\begin{Soutput}
[1]  0.2  6.1  9.3 -4.4  6.7
\end{Soutput}
\end{Schunk}
\vspace{20pt}
... kan als basis dienen voor een logische vector: 
\begin{Schunk}
\begin{Sinput}
> x <- numeric.v > 0
> print(x)
\end{Sinput}
\begin{Soutput}
[1]  TRUE  TRUE  TRUE FALSE  TRUE
\end{Soutput}
\end{Schunk}


\end{frame}


\end{document}
